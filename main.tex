\documentclass{beamer}
\usepackage[utf8]{inputenc}
\usepackage[french]{babel}
\usepackage{tikz}
\usepackage{xcolor}
\usetikzlibrary{shapes.misc, positioning}
\usepackage{pgfplots}
\usepackage{amsmath}
\usepackage{hyperref}
\usepackage{tabularx}
\usepackage{graphics}
\usepackage{multirow, multicol}
\DeclareMathOperator*{\argmax}{arg\,max}
\DeclareMathOperator*{\argmin}{arg\,min}


\title{Extraction de taxonomie par regroupement hiérarchique de plongements vectoriels de graphes de connaissances}
\author{Félix Martel}
\institute{Polytechnique Montréal}
\date{Août 2020}

% Theme
\usetheme[secheader]{Boadilla}
\setbeamertemplate{footline}[frame number]{}
\setbeamertemplate{navigation symbols}{}


\begin{document}

\frame{\titlepage}

  
 \AtBeginSubsection[]
  {
     \begin{frame}[noframenumbering]
    \frametitle{Plan}
    \tableofcontents[ 
        currentsection,
        currentsubsection,
        %hideothersubsections, 
        %sectionstyle=show/show, 
        %subsectionstyle=show/show, 
    ] 
  \end{frame}
  }

\section{Introduction}

\iffalse {

Introduction : graphes de connaissance, intérêt de l'extraction de taxonomies
Modèles de plongement
- présentation simple
- exemple de TransE ?
- ajouter slides finales pour autres modèles

Extraction non-expressive
- Idée générale
- Méthode MLI
- Méthode MLM
- Évaluation & discussion

Extraction expressive
- Limites de l'approche précédente
- Idée générale
- Plongement-regroupement
- Extraction d'axiomes
- Évaluation quanti
- Évaluation quali

}
\fi

% Introduction, définitions et enjeux
\begin{frame}{Introduction}
    Un graphe de connaissance est un graphe qui permet de stocker des données de manière \textbf{structurée}.

    \newline
    Il est composé d'\textcolor{blue}{entités} et de \textcolor{red}{relations}.

    \begin{figure}[H]
        \centering
        \begin{tikzpicture}
[   cnode/.style={draw=black,fill=#1,minimum width=3mm,circle},
    rnode/.style={draw=black,fill=#1,minimum width=6mm, minimum height=5mm, rectangle}
]
    

    \node[cnode=blue!20,draw=blue,label={[blue]180:\textit{Gaston Miron}}] (gm) at (0, 0) {};
    \node[cnode=gray!20, label=0:\textit{Poet}] (poet) at (3.5, 1) {};
    \node[cnode=gray!20, label=0:\textit{St-Agathe-des-Monts}] (sam) at (3.5, 0) {};
    \node[cnode=gray!20, label=0:\textit{8/01/1928}] (birth) at (3.5, -1) {};
    \node[cnode=gray!20, label=0:\textit{Person}] (sc) at (7, 2) {};
    
    \draw[->, red] (gm) to[bend left] node[midway, above, sloped]{\texttt{instanceOf}} (poet);
     \draw[->] (gm) -- node[midway, above]{\texttt{birthPlace}} (sam);
      \draw[->] (gm) to[bend right] node[midway, below, sloped]{\texttt{birthDate}} (birth);
       \draw[->] (poet) to[bend left] node[midway, above, sloped]{\texttt{subclassOf}} (sc);
    
\end{tikzpicture}
    \end{figure}
    
\end{frame}


\begin{frame}{Ontologie et taxonomie}

Un graphe de connaissances s'appuie sur une \textit{ontologie}, c'est-à-dire une série d'\textbf{axiomes logiques}, comme par exemple :
\begin{itemize}
    \item \textcolor{blue}{$\texttt{Athlete} \sqsubseteq \texttt{Person}$} : le type \texttt{Athlete} est un sous-type de \texttt{Person}, autrement dit un.e athlète est toujours une personne
    \item \textcolor{blue}{$\texttt{Poet} \equiv \texttt{Person} \sqcap \exists \texttt{is\_author\_of}.\texttt{Poetry}$} : un.e poète est une personne qui écrit de la poésie
    \item \textcolor{blue}{$\texttt{Agent} \equiv \texttt{Person} \sqcup \texttt{Organisation}$} : un agent est soit une personne, soit une organisation
    
\end{itemize}
\end{frame}

\begin{frame}{Ontologie et taxonomie}

Taxonomie : ontologie avec uniquement des axiomes de subsumption (\textit{«A est un sous-type de B»})

\begin{figure}[H]
    \centering
    \begin{tikzpicture}
[   cnode/.style={draw=black,fill=#1,minimum width=3mm,circle},
    rnode/.style={draw=gray!40,fill=gray!20,minimum width=6mm, minimum height=6mm, rectangle}
]
    

    \node[rnode] (l11) at (0, 0) {\texttt{Thing}};
    
    \node[rnode] (l21) at (-2, -1) {\texttt{Agent}};
    \node[rnode] (l22) at (0, -1) {\texttt{Place}};
    \node[rnode] (l23) at (2, -1) {\texttt{Event}};
    
    \node[rnode] (l31) at (-5, -2) {\texttt{Organisation}};
    \node[rnode] (l32) at (-2.5, -2) {\texttt{Person}};
    \node[rnode] (l33) at (-0.05, -2) {\texttt{NaturalPlace}};
    \node[rnode] (l34) at (3.1, -2) {\texttt{PopulatedPlace}};
    
    \node[rnode] (l41) at (-6, -3) {\texttt{Artist}};
    \node[rnode] (l42) at (-4.2, -3) {\texttt{Athlete}};
    \node[rnode] (l43) at (-2, -3) {\texttt{Politician}};
    \node[rnode] (l44) at (0.5, -3) {\texttt{Scientist}};
    
    \draw[->, gray!40, thick] (l11) -- (l21) {};
    \draw[->, gray!40, thick] (l11) -- (l22) {};
    \draw[->, gray!40, thick] (l21) -- (l31) {};
    \draw[->, gray!40, thick] (l21) -- (l32) {};
    \draw[->, gray!40, thick] (l22) -- (l33) {};
    \draw[->, gray!40, thick] (l22) -- (l34) {};
    \draw[->, gray!40, thick] (l32) -- (l41) {};
    \draw[->, gray!40, thick] (l32) -- (l42) {};
    \draw[->, gray!40, thick] (l32) -- (l43) {};
    \draw[->, gray!40, thick] (l32) -- (l44) {};
\end{tikzpicture}
\end{figure}
\end{frame}

\begin{frame}{Extraction de taxonomie}

\begin{columns}[t]
    \begin{column}{0.48\textwidth}
        %Content
        \textbf{Entrée} : 
        Un graphe de connaissance $\mathcal{KG}$ sans axiome de subsumption
        
        \begin{figure}
            \begin{tikzpicture}
                [   cnode/.style={draw=black,fill=#1,minimum width=3mm,circle},
            rnode/.style={draw=#1!40,fill=#1!20,minimum width=6mm, minimum height=6mm, rectangle},
            gnode/.style={draw=gray!40,fill=gray!20,circle,scale=0.5},
            gline/.style={gray!40}
        ]
        
        
        \node[gnode] (a1) at (0, 0) {};
        \node[gnode] (a2) at (0.8, 1.2) {};
        \node[gnode] (a3) at (0.5, 1.6) {};
        \node[gnode] (a4) at (0.1, 0.7) {};
        \node[gnode] (a5) at (1.5, 0.5) {};
        \node[gnode] (a6) at (1.3, 0.1) {};
        \node[gnode] (a7) at (2, 1.3) {};
        \node[gnode] (a8) at (1, 1) {};
        
        \draw[gline] (a1) -- (a2);
        \draw[gline] (a1) -- (a3);
        \draw[gline] (a1) -- (a8);
        \draw[gline] (a1) -- (a4);
        \draw[gline] (a7) -- (a6);
        \draw[gline] (a7) -- (a2);
        \draw[gline] (a2) -- (a3);
        \draw[gline] (a3) -- (a6);
        \draw[gline] (a4) -- (a6);
        \draw[gline] (a1) -- (a6);
        \draw[gline] (a7) -- (a8);
        \draw[gline] (a4) -- (a5);
        \draw[gline] (a3) -- (a5);
        \draw[gline] (a8) -- (a2);
            \end{tikzpicture}
        \end{figure}
    \end{column}
    \begin{column}{0.4\textwidth}
        %Content
        \textbf{Sortie} : un ensemble d'axiomes $t_k \sqsubseteq t_l$ qui forment une taxonomie des types
        \begin{figure}
            \begin{tikzpicture}
[   cnode/.style={draw=black,fill=#1,minimum width=3mm,circle},
    rnode/.style={draw=#1!40,fill=#1!20,minimum width=6mm, minimum height=6mm, rectangle},
    gnode/.style={draw=gray!40,fill=gray!20,circle,scale=0.5},
    gline/.style={gray!40}
]




\node[rnode=red] (a) at (5.8, 1.5) {$t_1$};
\node[rnode=blue] (b) at (5.3, 0.5) {$t_2$};
\node[rnode=purple] (c) at (6.3, 0.5) {$t_3$};

\node at (5.8, -0.5) {$t_2 \sqsubseteq t_1$};
\node at (5.8, -1) {$t_3 \sqsubseteq t_1$};
\draw[->] (a) -- (b) {};
\draw[->] (a) -- (c) {};
\node at (7.5, 0) {};
\end{tikzpicture}
        \end{figure}
    \end{column}
\end{columns}

\end{frame}

\begin{frame}{Modèles de plongement}
    But : obtenir une représentation vectorielle de chaque entité, telle que si deux entités sont sémantiquement proches, alors leurs plongements soient géométriquement proches.
    
    \begin{figure}
        \begin{tikzpicture}
    [   cnode/.style={draw=black,fill=#1,minimum width=3mm,circle},
        rnode/.style={draw=#1!40,fill=#1!20,minimum width=6mm, minimum height=6mm, rectangle}
    ]
    
    \draw [->] (0, 0) -- (0, 2);
    \draw [->] (0, 0) -- (2, 0);
    \draw [->] (0, 0) -- (-0.5, -1.4);
    
    \draw [red] plot [only marks, draw=red, mark=*, mark size=0.5, domain=0:2, samples=300] ({4*(rnd-0.5)*rnd-0.3},{4*(rnd-0.5)*rnd-0.2});
    
    \draw [blue] plot [only marks, draw=red, mark=*, mark size=0.5, domain=0:2, samples=100] ({3*(rnd-0.5)*rnd+0.1},{4*(rnd-0.5)*rnd+0.1});
    
    \draw [purple] plot [only marks, draw=red, mark=*, mark size=0.5, domain=0:2, samples=300] ({1*(rnd-0.5)*rnd+0.5},{1*(rnd-0.5)*rnd-0.4});
    
    \node [red] at (-1.5, -1.5) {$t_1$};
    \node [blue] at (1.7, 1) {$t_2$};
    \node [purple] at (1.5, -1.5) {$t_3$};
    \end{tikzpicture}
    \end{figure}
    
\end{frame}

\begin{frame}{Modèles de plongement : un exemple}
    Un exemple d'algorithme de plongement : \textbf{TransE} \cite{bordes2013translating}.
    \newline \newline
    TransE considère une relation $r$ comme une translation entre deux entités.

    %\begin{figure}[H]
    %    \centering
    %    \begin{tikzpicture}
[   cnode/.style={draw=black,fill=#1,minimum width=3mm,circle},
    rnode/.style={draw=black,fill=#1,minimum width=6mm, minimum height=5mm, rectangle}
]
    

    \node[cnode=blue!20] (h) at (0, 0) {$h$};
    \node[cnode=red!20] (t) at (3.5, 0) {$t$};
    
    \draw[->] (h) to[bend left] node[midway, above]{$r$} (t);
    
\end{tikzpicture}
    %\end{figure}
    \begin{figure}[H]
        \centering
        \begin{tikzpicture}
[   cnode/.style={draw=black,fill=#1,minimum width=3mm,circle},
    rnode/.style={draw=black,fill=#1,minimum width=6mm, minimum height=5mm, rectangle}
]
    
    \draw[->,blue,thick] (0,0) -- node[above, sloped]{$\mathbf{h}$} (1, 1.8);
    \draw[->,red, thick] (0,0) -- node[above, sloped]{$\mathbf{t}$} (3, 1.2);
    \draw[->] (1,1.8) -- node[above, sloped]{$\mathbf{r}$} (2.9, 1.25);
    
    \draw[->] (0,0) -- (4,0);
    \draw[->] (0,0) -- (-0.5, -0.75);
    \draw[->] (0,0) -- (0,2);
    
    \node at (2, -1.5) {$(h, r, t) \in \mathcal{KG} \iff \mathbf{h} + \mathbf{r} \approx \mathbf{t} $};
\end{tikzpicture}
    \end{figure}
\end{frame}

\section{Extraction de taxonomie}

\subsection{Idée générale}

\begin{frame}{Idée générale}
    \begin{figure}
        \begin{tikzpicture}
[   cnode/.style={draw=gray!40,fill=gray!20, minimum width=3mm, circle,inner sep=0.06, scale=0.8},
    colnode/.style={fill=#1,minimum width=3mm,circle,inner sep=0.06, scale=0.8},
    rnode/.style={draw=#1!40,fill=#1!20,minimum width=6mm, minimum height=6mm, rectangle},
    cline/.style={gray!40,thick},
    recnode/.style={fill=#1,minimum width=3mm, minimum height=3mm, rectangle, scale=0.8}
]

%
\node at (0, 4.5) {Données};
\node at (0, 4) {$\mathcal{KG}+\{(e_i, t_i)\}$};

\draw [->] (0, 0) -- (0, 2);
\draw [->] (0, 0) -- (1.3, 0);
\draw [->] (0, 0) -- (-0.5, -1.4);

\draw [red] plot [only marks, draw=red, mark=*, mark size=0.5, domain=0:2, samples=120] ({2*(rnd-0.5)*rnd-0.3},{4*(rnd-0.5)*rnd-0.2});

\draw [blue] plot [only marks, draw=red, mark=*, mark size=0.5, domain=0:2, samples=50] ({(rnd-0.5)*rnd+0.1},{4*(rnd-0.5)*rnd+0.1});

\draw [purple] plot [only marks, draw=red, mark=*, mark size=0.5, domain=0:2, samples=50] ({1*(rnd-0.5)*rnd+0.5},{1*(rnd-0.5)*rnd-0.4});

%\node [red] at (-1.5, -1.5) {$t_1$};
%\node [blue] at (1.7, 1) {$t_2$};
%\node [purple] at (1.5, -1.5) {$t_3$};

    \draw[->, double] (0, 3.5) to[right] node {1} (0, 2.5);
    \draw[->, double] (1.5, -0.5) to[above] node {2} (2.5, -0.5);
    \draw[->, double] (5.5, -0.5) to[above] node {3} (6.5, -0.5);
    \draw[->, double] (8, 1) to[left] node {4} (8, 2); 

    % CLUSTERING %
    \def\offsetxa{4};
    \def\stepx{0.4};
    \def\stepy{0.7};
    
    \node[cnode] (000) at (-0.5*\stepx+\offsetxa,-2*\stepy) {};
    \node[cnode] (001) at (-1.5*\stepx+\offsetxa,-2*\stepy) {};
    \node[cnode] (010) at (-2.5*\stepx+\offsetxa,-2*\stepy) {};
    \node[cnode] (011) at (-3.5*\stepx+\offsetxa,-2*\stepy) {};
    \node[cnode] (100) at (0.5*\stepx+\offsetxa,-2*\stepy) {};
    \node[cnode] (101) at (1.5*\stepx+\offsetxa,-2*\stepy) {};
    \node[cnode] (110) at (2.5*\stepx+\offsetxa,-2*\stepy) {};
    \node[cnode] (111) at (3.5*\stepx+\offsetxa,-2*\stepy) {};
    
    \node[cnode] (00) at (-3*\stepx+\offsetxa,-\stepy) {};
    \node[cnode] (01) at (-1*\stepx+\offsetxa,-\stepy) {};
    \node[cnode] (10) at (1*\stepx+\offsetxa,-\stepy) {};
    \node[cnode] (11) at (3*\stepx+\offsetxa,-\stepy) {};
    
    \node[cnode] (0) at (-2*\stepx+\offsetxa,0*\stepy) {};
    \node[cnode] (1) at (2*\stepx+\offsetxa,0*\stepy) {};
    
    \node[cnode] (root) at (\offsetxa,\stepy) {};
    
    \draw[cline] (root) -- (0) {};
    \draw[cline] (root) -- (1) {};
    
    \draw[cline] (00) -- (0) {};
    \draw[cline] (10) -- (1) {};
    \draw[cline] (01) -- (0) {};
    \draw[cline] (11) -- (1) {};
    
    \draw[cline] (01) -- (000) {};
    \draw[cline] (01) -- (001) {};
    \draw[cline] (00) -- (010) {};
    \draw[cline] (00) -- (011) {};
    \draw[cline] (10) -- (100) {};
    \draw[cline] (10) -- (101) {};
    \draw[cline] (11) -- (110) {};
    \draw[cline] (11) -- (111) {};
    
    % CLUSTERING 2 %
    \def\offsetxc{\offsetxa+4};
    \def\stepx{0.4};
    \def\stepy{0.7};
    
    \node[colnode=blue!70] (000) at (-0.5*\stepx+\offsetxc,-2*\stepy) {};
    \node[cnode] (001) at (-1.5*\stepx+\offsetxc,-2*\stepy) {};
    \node[colnode=blue!60] (010) at (-2.5*\stepx+\offsetxc,-2*\stepy) {};
    \node[colnode=blue!80] (011) at (-3.5*\stepx+\offsetxc,-2*\stepy) {};
    \node[cnode] (100) at (0.5*\stepx+\offsetxc,-2*\stepy) {};
    \node[colnode=blue!50] (101) at (1.5*\stepx+\offsetxc,-2*\stepy) {};
    \node[colnode=purple!60] (110) at (2.5*\stepx+\offsetxc,-2*\stepy) {};
    \node[cnode] (111) at (3.5*\stepx+\offsetxc,-2*\stepy) {};
    
    \node[cnode] (00) at (-3*\stepx+\offsetxc,-\stepy) {};
    \node[cnode] (01) at (-1*\stepx+\offsetxc,-\stepy) {};
    \node[colnode=purple!70] (10) at (1*\stepx+\offsetxc,-\stepy) {};
    \node[cnode] (11) at (3*\stepx+\offsetxc,-\stepy) {};
    
    \node[colnode=purple!80] (0) at (-2*\stepx+\offsetxc,0*\stepy) {};
    \node[cnode] (1) at (2*\stepx+\offsetxc,0*\stepy) {};
    
    \node[colnode=red!80] (root) at (\offsetxc,\stepy) {};
    
    \draw[cline] (root) -- (0) {};
    \draw[cline] (root) -- (1) {};
    
    \draw[cline] (00) -- (0) {};
    \draw[cline] (10) -- (1) {};
    \draw[cline] (01) -- (0) {};
    \draw[cline] (11) -- (1) {};
    
    \draw[cline] (01) -- (000) {};
    \draw[cline] (01) -- (001) {};
    \draw[cline] (00) -- (010) {};
    \draw[cline] (00) -- (011) {};
    \draw[cline] (10) -- (100) {};
    \draw[cline] (10) -- (101) {};
    \draw[cline] (11) -- (110) {};
    \draw[cline] (11) -- (111) {};

%\node at (4, 0) {\Large $\Longrightarrow$};

% TAXONOMY
    \def\offsety{4};
    \def\offsetxtaxo{\offsetxc};

    \def\skipx{0.7};
    \def\skipy{0.7};
    
    \node[recnode=red!80] (l11) at (\offsetxtaxo, \offsety) {};
    
    \node[recnode=purple!80] (l21) at (-1*\skipx+\offsetxtaxo, \offsety-1*\skipy) {};
    \node[recnode=purple!70] (l22) at (\offsetxtaxo, \offsety-1*\skipy) {};
    \node[recnode=purple!60] (l23) at (1*\skipx+\offsetxtaxo, \offsety-1*\skipy) {};
    
    \node[recnode=blue!80] (l31) at (-3*\skipx+\offsetxtaxo, \offsety-2*\skipy) {};
    \node[recnode=blue!70] (l32) at (-2*\skipx+\offsetxtaxo, \offsety-2*\skipy) {};
    \node[recnode=blue!60] (l33) at (-1*\skipx+\offsetxtaxo, \offsety-2*\skipy) {};
    \node[recnode=blue!50] (l34) at (\offsetxtaxo, \offsety-2*\skipy) {};
    
    \draw[->, gray!40, thick] (l11) -- (l21) {};
    \draw[->, gray!40, thick] (l11) -- (l22) {};
    \draw[->, gray!40, thick] (l11) -- (l23) {};
    \draw[->, gray!40, thick] (l21) -- (l31) {};
    \draw[->, gray!40, thick] (l21) -- (l32) {};
    \draw[->, gray!40, thick] (l21) -- (l33) {};
    \draw[->, gray!40, thick] (l22) -- (l34) {};
\end{tikzpicture}
    \end{figure}
\end{frame}


\subsection{Méthode par liaison injective}


\begin{frame}{Regroupement hiérarchique}
On applique un algorithme de \textit{clustering} hiérarchique aux plongements $\mathbf{e}_1, \ldots, \mathbf{e}_N$ :
\begin{figure}
    \begin{tikzpicture}
[   cnode/.style={draw=gray!40,fill=gray!20,minimum width=0.2mm,circle},
    cline/.style={gray!40,thick}
]

\def\stepx{0.7};
\def\stepy{0.5};

\node[cnode] (000) at (-0.5*\stepx,-2*\stepy) {};
\node[cnode] (001) at (-1.5*\stepx,-2*\stepy) {};
\node[cnode] (010) at (-2.5*\stepx,-2*\stepy) {};
\node[cnode] (011) at (-3.5*\stepx,-2*\stepy) {};
\node[cnode] (100) at (0.5*\stepx,-2*\stepy) {};
\node[cnode] (101) at (1.5*\stepx,-2*\stepy) {};
\node[cnode] (110) at (2.5*\stepx,-2*\stepy) {};
\node[cnode] (111) at (3.5*\stepx,-2*\stepy) {};

\node[cnode] (00) at (-3*\stepx,-\stepy) {};
\node[cnode] (01) at (-1*\stepx,-\stepy) {};
\node[cnode] (10) at (1*\stepx,-\stepy) {};
\node[cnode] (11) at (3*\stepx,-\stepy) {};

\node[cnode] (0) at (-2*\stepx,0*\stepy) {};
\node[cnode] (1) at (2*\stepx,0*\stepy) {};

\node[cnode, label=90:{$\{e_1, \ldots, e_N\}$}] (root) at (0,\stepy) {};

\node[cnode, label=270:{$e_1$}] (t1) at (-4*\stepx,-4*\stepy) {};
\node[cnode, label=270:{$e_2$}] (t2) at (-3.2*\stepx,-4*\stepy) {};
\node[label=270:{$\ldots$}] at (0.4*\stepx, -4*\stepy) {};
\node[cnode, label=270:{$e_N$}] (tn) at (4*\stepx,-4*\stepy) {};

\draw[cline] (root) -- (0) {};
\draw[cline] (root) -- (1) {};

\draw[cline] (00) -- (0) {};
\draw[cline] (10) -- (1) {};
\draw[cline] (01) -- (0) {};
\draw[cline] (11) -- (1) {};

\draw[cline] (01) -- (000) {};
\draw[cline] (01) -- (001) {};
\draw[cline] (00) -- (010) {};
\draw[cline] (00) -- (011) {};
\draw[cline] (10) -- (100) {};
\draw[cline] (10) -- (101) {};
\draw[cline] (11) -- (110) {};
\draw[cline] (11) -- (111) {};

\draw[dashed,gray!40] (011) -- (t1) {};
\draw[dashed,gray!40] (111) -- (tn) {};

\end{tikzpicture}
\end{figure}
\end{frame}

\begin{frame}{Association type-cluster}
    \begin{figure}
        \begin{tikzpicture}
            [   cnode/.style={fill=blue, circle},
            ]
            \def\x{0.5};
            \def\recx{-0.5};
            \node [cnode] at (0, 0) {};
            \node [cnode] at (1*\x, 0) {};
            \node [cnode] at (2*\x, 0) {};
            \node [fill=red, circle] at (3*\x, 0) {};
            \node [cnode] at (5*\x, 0) {};
            \node [cnode] at (6*\x, 0) {};
            \draw[rounded corners] (-\x, -\x) rectangle (4*\x, \x) {};
            \node [] at (-1.5, 0) {cluster $C$};  
        \end{tikzpicture}
    \end{figure}

Pour un cluster $C$ et un type $t$, on peut calculer :
\begin{itemize}
    \item la \textbf{précision}, soit la proportion d'éléments de $C$ qui sont de type $t$ :
    $$p(C, \textcolor{blue}{\bullet}) = \frac{3}{4}$$
    \item le \textbf{rappel}, soit la proportion d'éléments de type $t$ qui font partie de $C$
    $$r(C, \textcolor{blue}{\bullet}) = \frac{3}{5}$$
    \item la \textbf{mesure $F_1$}, qui est la moyenne harmonique de la précision et du rappel
    $$F(C, \textcolor{blue}{\bullet}) = 2(\frac{1}{p} + \frac{1}{r})^{-1} = \textcolor{blue}{\frac{2}{3}}$$
\end{itemize}
\end{frame}

\begin{frame}{Liaison injective optimale}
    On veut associer à chaque type $t$ un cluster unique $m(t)$. Pour cela, on choisit l'assignation $t \leftarrow m(t)$ qui maximise la mesure $F_1$ globale :
    $$m^* = \argmax_{m \only<5>{\text{ injective}}} \sum_t F(m(t), t)$$
    \begin{figure}
        \begin{tikzpicture}
[   cnode/.style={draw=gray!40,fill=gray!20,minimum width=0.2mm,circle},
    colnode/.style={draw=#1!40,fill=#1!20,minimum width=0.2mm,circle},
    cline/.style={gray!40,thick}
]

\def\stepx{0.7};
\def\stepy{0.5};

\node[cnode] (000) at (-0.5*\stepx,-2*\stepy) {};
\node[cnode] (001) at (-1.5*\stepx,-2*\stepy) {};
\node[cnode] (010) at (-2.5*\stepx,-2*\stepy) {};
\node[cnode] (011) at (-3.5*\stepx,-2*\stepy) {};
\node[cnode] (100) at (0.5*\stepx,-2*\stepy) {};
\node[cnode] (101) at (1.5*\stepx,-2*\stepy) {};
\node[cnode] (110) at (2.5*\stepx,-2*\stepy) {};
\node[cnode] (111) at (3.5*\stepx,-2*\stepy) {};

\node[cnode] (00) at (-3*\stepx,-\stepy) {};
\node[cnode] (01) at (-1*\stepx,-\stepy) {};
\node[cnode] (10) at (1*\stepx,-\stepy) {};
\node[cnode] (11) at (3*\stepx,-\stepy) {};

\node[cnode] (0) at (-2*\stepx,0*\stepy) {};
\node[cnode] (1) at (2*\stepx,0*\stepy) {};

\node[cnode, label=90:{$\{e_1, \ldots, e_N\}$}] (root) at (0,\stepy) {};

\def\minx{-5.5*\stepx};

\node (e1) at (\minx, 0) {$t_1$};
\node (e2) at (\minx, -\stepy) {$t_2$};
\node at (\minx, -2.5*\stepy) {$\vdots$};
\node (em) at (\minx, -4*\stepy) {$t_M$};

\draw<2->[->, red] (e1) to[bend left] node[midway, above, sloped]{$m^*(t_1)$} (0);
\draw<3->[->, blue] (e2) to[bend right] node[midway, below, sloped]{$m^*(t_2)$} (110);


\node[cnode, label=270:{$e_1$}] (t1) at (-4*\stepx,-4*\stepy) {};
\node[cnode, label=270:{$e_2$}] (t2) at (-3.2*\stepx,-4*\stepy) {};
\node[label=270:{$\ldots$}] at (0.4*\stepx, -4*\stepy) {};
\node[cnode, label=270:{$e_N$}] (tn) at (4*\stepx,-4*\stepy) {};

\draw[cline] (root) -- (0) {};
\draw[cline] (root) -- (1) {};

\draw[cline] (00) -- (0) {};
\draw[cline] (10) -- (1) {};
\draw[cline] (01) -- (0) {};
\draw[cline] (11) -- (1) {};

\draw[cline] (01) -- (000) {};
\draw[cline] (01) -- (001) {};
\draw[cline] (00) -- (010) {};
\draw[cline] (00) -- (011) {};
\draw[cline] (10) -- (100) {};
\draw[cline] (10) -- (101) {};
\draw[cline] (11) -- (110) {};
\draw[cline] (11) -- (111) {};

\draw[dashed,gray!40] (011) -- (t1) {};
\draw[dashed,gray!40] (111) -- (tn) {};

    

\end{tikzpicture}
    \end{figure}
\end{frame}

\begin{frame}{Extraction de la taxonomie}
    On peut maintenant extraire le sous-arbre composé uniquement des clusters choisis et trouver les axiomes:
    \begin{figure}
        \begin{tikzpicture}
    [   cnode/.style={draw=gray!40,fill=gray!20,minimum width=0.2mm,circle},
        colnode/.style={draw=red!80,fill=red!40,minimum width=0.2mm,circle},
        cline/.style={gray!40,thick},
    rnode/.style={draw=gray!40,fill=gray!20,minimum width=6mm, minimum height=6mm, rectangle}
    ]
    
    \def\stepx{0.7};
    \def\stepy{0.5};
    
    \node[colnode, label=270:$t_3$] (000) at (-0.5*\stepx,-2*\stepy) {};
    \node[colnode, label=270:$t_4$] (001) at (-1.5*\stepx,-2*\stepy) {};
    \node[cnode] (010) at (-2.5*\stepx,-2*\stepy) {};
    \node[cnode] (011) at (-3.5*\stepx,-2*\stepy) {};
    \node[cnode] (100) at (0.5*\stepx,-2*\stepy) {};
    \node[colnode, label=270:$t_6$] (101) at (1.5*\stepx,-2*\stepy) {};
    \node[cnode] (110) at (2.5*\stepx,-2*\stepy) {};
    \node[colnode, label=270:$t_7$] (111) at (3.5*\stepx,-2*\stepy) {};
    
    \node[colnode, label=110:$t_2$] (00) at (-3*\stepx,-\stepy) {};
    \node[cnode] (01) at (-1*\stepx,-\stepy) {};
    \node[colnode, label=110:$t_5$] (10) at (1*\stepx,-\stepy) {};
    \node[cnode] (11) at (3*\stepx,-\stepy) {};
    
    \node[colnode, label=90:$t_1$] (0) at (-2*\stepx,0*\stepy) {};
    \node[cnode] (1) at (2*\stepx,0*\stepy) {};
    
    \node[cnode] (root) at (0,\stepy) {};
    
    \def\minx{-5.5*\stepx};
    
    \node[cnode] (t1) at (-4*\stepx,-4*\stepy) {};
    \node[cnode] (t2) at (-3.2*\stepx,-4*\stepy) {};
    \node[cnode] (tn) at (4*\stepx,-4*\stepy) {};
    
    \draw[cline] (root) -- (0) {};
    \draw[cline] (root) -- (1) {};
    
    \draw[cline] (00) -- (0) {};
    \draw[cline] (10) -- (1) {};
    \draw[cline] (01) -- (0) {};
    \draw[cline] (11) -- (1) {};
    
    \draw[cline] (01) -- (000) {};
    \draw[cline] (01) -- (001) {};
    \draw[cline] (00) -- (010) {};
    \draw[cline] (00) -- (011) {};
    \draw[cline] (10) -- (100) {};
    \draw[cline] (10) -- (101) {};
    \draw[cline] (11) -- (110) {};
    \draw[cline] (11) -- (111) {};
    
    \draw[dashed,gray!40] (011) -- (t1) {};
    \draw[dashed,gray!40] (111) -- (tn) {};
    
    \def\offsety{-6*\stepy};

    \node[rnode] (l11) at (0, \offsety) {\texttt{root}};
    
    \node[rnode] (l21) at (-1, \offsety-1) {$t_1$};
    \node[rnode] (l22) at (0, \offsety-1) {$t_5$};
    \node[rnode] (l23) at (1, \offsety-1) {$t_7$};
    
    \node[rnode] (l31) at (-3, \offsety-2) {$t_2$};
    \node[rnode] (l32) at (-2, \offsety-2) {$t_3$};
    \node[rnode] (l33) at (-1, \offsety-2) {$t_4$};
    \node[rnode] (l34) at (0, \offsety-2) {$t_6$};
    
    \draw[->, gray!40, thick] (l11) -- (l21) {};
    \draw[->, gray!40, thick] (l11) -- (l22) {};
    \draw[->, gray!40, thick] (l11) -- (l23) {};
    \draw[->, gray!40, thick] (l21) -- (l31) {};
    \draw[->, gray!40, thick] (l21) -- (l32) {};
    \draw[->, gray!40, thick] (l21) -- (l33) {};
    \draw[->, gray!40, thick] (l22) -- (l34) {};
    \end{tikzpicture}
    \end{figure}
\end{frame}

\subsection{Méthode par liaison multiple}

\begin{frame}{Méthode par liaison multiple (MLM)}
    Se base également sur un regroupement hiérarchique.
    
    \textbf{Idée :} remplacer la liaison entre un type $t$ et un unique cluster $m^*(t)$ en une \textbf{probabilité de liaison} $P(Z_t = C)$ entre un type $t$ et un cluster $C$.
    \begin{figure}
        \begin{tikzpicture}
[   cnode/.style={draw=gray!40,fill=gray!20,minimum width=0.2mm,circle},
    colnode/.style={draw=#1!40,fill=#1!20,minimum width=0.2mm,circle},
    cline/.style={gray!40,thick}
]

\def\stepx{0.7};
\def\stepy{0.5};

\node[cnode] (000) at (-0.5*\stepx,-2*\stepy) {};
\node[cnode] (001) at (-1.5*\stepx,-2*\stepy) {};
\node[cnode] (010) at (-2.5*\stepx,-2*\stepy) {};
\node[cnode] (011) at (-3.5*\stepx,-2*\stepy) {};
\node[cnode] (100) at (0.5*\stepx,-2*\stepy) {};
\node[cnode] (101) at (1.5*\stepx,-2*\stepy) {};
\node[cnode] (110) at (2.5*\stepx,-2*\stepy) {};
\node[cnode] (111) at (3.5*\stepx,-2*\stepy) {};

\node[cnode] (00) at (-3*\stepx,-\stepy) {};
\node[cnode] (01) at (-1*\stepx,-\stepy) {};
\node[cnode] (10) at (1*\stepx,-\stepy) {};
\node[cnode] (11) at (3*\stepx,-\stepy) {};

\node[cnode] (0) at (-2*\stepx,0*\stepy) {};
\node[cnode] (1) at (2*\stepx,0*\stepy) {};

 \node[cnode] (root) at (0,\stepy) {};

\def\minx{-5.5*\stepx};

\node (e1) at (\minx, 0) {$t_1$};
\node (e2) at (\minx, -\stepy) {$t_2$};
\node at (\minx, -2.5*\stepy) {$\vdots$};
\node (em) at (\minx, -4*\stepy) {$t_M$};



\node[cnode] (t1) at (-4*\stepx,-4*\stepy) {};
\node[cnode] (t2) at (-3.2*\stepx,-4*\stepy) {};
%\node[label=270:{$\ldots$}] at (0.4*\stepx, -4*\stepy) {};
\node[cnode] (tn) at (4*\stepx,-4*\stepy) {};

\draw[cline] (root) -- (0) {};
\draw[cline] (root) -- (1) {};

\draw[cline] (00) -- (0) {};
\draw[cline] (10) -- (1) {};
\draw[cline] (01) -- (0) {};
\draw[cline] (11) -- (1) {};

\draw[cline] (01) -- (000) {};
\draw[cline] (01) -- (001) {};
\draw[cline] (00) -- (010) {};
\draw[cline] (00) -- (011) {};
\draw[cline] (10) -- (100) {};
\draw[cline] (10) -- (101) {};
\draw[cline] (11) -- (110) {};
\draw[cline] (11) -- (111) {};

\draw[dashed,gray!40] (011) -- (t1) {};
\draw[dashed,gray!40] (111) -- (tn) {};

\draw<1>[->, red] (e1) to[bend left] node[midway, above, sloped]{$m^*(t_1)$} (0);
\draw<1>[->, blue] (e2) to[bend right] node[midway, below, sloped]{$m^*(t_2)$} (110);

\draw<2>[->, ultra thick, red] (e1.east) to [bend left] node[midway, above] {$Z_{t_1}$} (0);
\draw<2>[->, thick, red] (e1.east) to [bend left] (00);
\draw<2>[->, ultra thin, red] (e1.east) to [bend left] (01);
\draw<2>[->, red] (e1.east) to [bend right] (000);
\draw<2>[->, thin, red] (e1.east) to [bend right] (011);

\draw<3>[->, ultra thick, blue] (e2.east) to [bend left] (110);
\draw<3>[->, blue] (e2.east) to [bend left] node[midway, above] {$Z_{t_2}$} (1);
\draw<3>[->, thick, blue] (e2.east) to [bend left] (11);
\draw<3>[->, ultra thin, blue] (e2.east) to [bend right] (111);
\draw<3>[->, thin, blue] (e2.east) to [bend right] (100);
    

\end{tikzpicture}
    \end{figure}
\end{frame}

\begin{frame}{Méthode par liaison multiple (MLM)}
    Les mesures $F_1$ sont transformées en probabilités avec un softmax :
    \begin{equation}
        P(Z_t = C) = \frac{\displaystyle e^{ \beta F_1(t, C)}}{\displaystyle \sum_{C' \in \mathcal{C}} e^{\beta F_1(t, C')}}
    \end{equation}
    L'axiome $t' \sqsubset t$ devient un événement aléatoire $Z_{t'} \subset Z_{t}$.
    \begin{align}
        P(t' \sqsubset t) &= P(Z_{t'} \subset Z_{t}) \\
        &= \sum_{\substack{C, C' \in \mathcal{C} \\ C' \subset C}} P(Z_{t }= C) \times P(Z_{t'} = C')
    \end{align}
\end{frame}

\begin{frame}{MLM : exemple de calcul}
    \begin{figure}
        \begin{tikzpicture}[
  box/.style={draw=gray, thick, align=center},
  arrow/.style={->, shorten >=3pt, gray!80},
  node/.style={draw=gray, fill=gray!20},
  hnode/.style={draw=red, fill=red!20},
  snode/.style={draw=red, fill=red!20, very thick},
  point/.style={circle, draw=gray!80, fill=gray!30, node distance=1cm and 0.5cm, inner sep=0, minimum height=0.25cm},
  spoint/.style={circle, draw=gray!80, fill=gray!30, node distance=1cm and 0.3cm, inner sep=0, minimum height=0.25cm},
  pointA/.style={fill=red, draw=red},
  pointB/.style={fill=blue, draw=blue},
]

\def\y{1.5cm};
\def\x{0.5cm};

\begin{scope}
\node[point] (root) at (0, 0) {};
\node[point, below left=of root] (a1) {};
\node[point, below right=of root] (a2) {};
\node[point, below left=of a1] (b1) {};
\node[point, below left=of a2] (b2) {};

\node[red, left=1.5cm of a1] (A) {$A$};
\node[blue, right=1.5cm of a2] (B) {$B$};

\draw[arrow] (root) to (a1);
\draw[arrow] (root) to (a2);
\draw[arrow] (a1) to (b1);
\draw[arrow] (a1) to (b2);

\draw<2, 4-7, 9>[->, red] (A) to [bend left] node[midway, above, sloped] {$0,3$} (root);
\draw<2, 4, 8, 9>[->, thick, red] (A) to [bend left] node[midway, above, sloped] {$0,5$} (a1);
\draw<2, 4, 9>[->, thin, red] (A) to [bend right] node[midway, below, sloped] {$0,1$} (b1);
\draw<2, 4, 9>[->, thin, red] (A) to [bend right] node[midway, above, sloped] {$0,1$} (b2);

\draw<3-4, 7, 9>[->, blue] (B) to [bend right] node[midway, below, sloped] {$0,3$} (a2);
\draw<3-4, 5, 9>[->, thick, blue] (B) to [bend right] node[midway, above, sloped] {$0,5$} (a1);
\draw<3-4, 6, 8, 9>[->, blue] (B) to [bend left] node[midway, below, sloped] {$0,2$} (b2);

\end{scope}


%\node at (0, -4cm) {$B \sqsubset A \iff Z_B \subset Z_A$ : 4 cas possibles};

\node<4> at (0, -4cm) {Calcul de $P(B \sqsubset A)$ ?};

\def\shift{1.5cm}
\def\shifty{-3.2cm}



\begin{scope}[yshift=\shifty, xshift=-3*\shift]
\node<5->[spoint, pointA, label={[red]180:{$Z_A$}}] (root) at (0, 0) {};
\node<5->[spoint, pointB, below left=of root, label={[blue]180:{$Z_B$}}] (a1) {};
\node<5->[spoint, below right=of root] (a2) {};
\node<5->[spoint, below left=of a1] (b1) {};
\node<5->[spoint, below left=of a2] (b3) {};
\node<5->[spoint, below right=of a1] (b2) {};
\draw<5->[arrow] (root) to (a1);
\draw<5->[arrow] (root) to (a2);
\draw<5->[arrow] (a1) to (b1);
\draw<5->[arrow] (a1) to (b2);
\node<5->[below=0.3cm of b2] (eq)  {\textcolor{red}{$0,3$} $\times$ \textcolor{blue}{$0,5$}};
\node<5->[right=0.3cm of eq] {$+$};
\end{scope}


\begin{scope}[yshift=\shifty, xshift=-\shift]
\node<6->[spoint, pointA, label={[red]180:{$Z_A$}}] (root) at (0, 0) {};
\node<6->[spoint, below left=of root] (a1) {};
\node<6->[spoint, below right=of root] (a2) {};
\node<6->[spoint, below left=of a1] (b1) {};
\node<6->[spoint, below left=of a2] (b3) {};
\node<6->[spoint, pointB, below right=of a1, label={[blue]0:{$Z_B$}}] (b2) {};
\draw<6->[arrow] (root) to (a1);
\draw<6->[arrow] (root) to (a2);
\draw<6->[arrow] (a1) to (b1);
\draw<6->[arrow] (a1) to (b2);
\node<6->[below=0.3cm of b2] (eq)  {\textcolor{red}{$0,3$} $\times$ \textcolor{blue}{$0,2$}};
\node<6->[right=0.3cm of eq] {$+$};
\end{scope}


\begin{scope}[yshift=\shifty, xshift=\shift]
\node<7->[spoint, pointA, label={[red]180:{$Z_A$}}] (root) at (0, 0) {};
\node<7->[spoint, below left=of root] (a1) {};
\node<7->[spoint, pointB, below right=of root, label={[blue]270:{$Z_B$}}] (a2) {};
\node<7->[spoint, below left=of a1] (b1) {};
\node<7->[spoint, below left=of a2] (b3) {};
\node<7->[spoint, below right=of a1] (b2) {};
\draw<7->[arrow] (root) to (a1);
\draw<7->[arrow] (root) to (a2);
\draw<7->[arrow] (a1) to (b1);
\draw<7->[arrow] (a1) to (b2);
\node<7->[below=0.3cm of b2] (eq)  {\textcolor{red}{$0,3$} $\times$ \textcolor{blue}{$0,3$}};
\node<7->[right=0.3cm of eq] {$+$};

\end{scope}


\begin{scope}[yshift=\shifty, xshift=3*\shift]
\node<8->[spoint] (root) at (0, 0) {};
\node<8->[spoint, pointA, below left=of root, label={[red]180:{$Z_A$}}] (a1) {};
\node<8->[spoint, below right=of root] (a2) {};
\node<8->[spoint, below left=of a1] (b1) {};
\node<8->[spoint, below left=of a2] (b3) {};
\node<8->[spoint, pointB, below right=of a1, label={[blue]0:{$Z_B$}}] (b2) {};
\draw<8->[arrow] (root) to (a1);
\draw<8->[arrow] (root) to (a2);
\draw<8->[arrow] (a1) to (b1);
\draw<8->[arrow] (a1) to (b2);
\node<8->[below=0.3cm of b2] (eq)  {\textcolor{red}{$0,5$} $\times$ \textcolor{blue}{$0,2$}};
\end{scope}

\node<9->[draw=black, inner sep=0.1cm] at (0, -7cm) {$P(B \sqsubset A) = \sum_{\substack{C, C' \in \mathcal{C} \\ C' \subset C}} P(Z_A= C)\cdot P(Z_B = C')  = 0,4$};

\end{tikzpicture}
    \end{figure}
\end{frame}

\begin{frame}{MLM : extraction de taxonomie}
\begin{enumerate}
    \item Construction d'un DAG $G$ (\textit{graphe orienté acyclique})

\begin{itemize}
    \item On part d'un graphe $G$ vide
    \item Pour chaque axiome $\alpha = t' \sqsubset t$ (trié par ordre de probabilité décroissante) :
    \begin{itemize}
        \item Si ajouter $\alpha$ à $G$ crée un cycle, on le rejette
        \item Sinon, on ajoute $\alpha$ à $G$ (\textit{i.e} ajout d'une arête orientée entre $t$ et $t'$)
    \end{itemize}
    \item On poursuit tant que la probabilité de $\alpha$ est supérieure à un seuil $\delta$
\end{itemize}
    \item Calcul de la réduction transitive $G_\text{red}$
    \item Suppression des parents multiples
\end{enumerate}
    \begin{figure}
        \begin{tikzpicture}[
  box/.style={draw=gray, thick, align=center},
  arrow/.style={->, shorten >=3pt},
  node/.style={draw=gray, fill=gray!20},
  hnode/.style={draw=red, fill=red!20},
  snode/.style={draw=red, fill=red!20, very thick},
  point/.style={circle, draw=blue, fill=blue, node distance=0.7cm and 0.3cm, inner sep=0, minimum height=0.1cm},
  verysure/.style={very thick},
  sure/.style={thick},
  unsure/.style={line width=0.01cm, draw=black!50},
  medsure/.style={draw=black!80}
]

\def\y{0.5cm};
\def\x{0.3cm};

\begin{scope}

\node[point] (root) at (0, 0) {};
\node[point, below left=of root] (a1) {};
\node[point, below right=of root] (a2) {};
\node[point, below left=of a1] (b1) {};
\node[point, below left=of a2] (b2) {};
\node[point, below right=of a2] (b3) {};
\node[point, below left=of b3] (c1) {};
\node[point, below=of b3] (c2) {};
\node[point, below right=of b3] (c3) {};

\draw[arrow, verysure] (root) to (a1);
\draw[arrow, sure] (root) to (a2);
\draw[arrow, unsure] (root) to (b2);
\draw[arrow, medsure] (root) to[bend left] (b3);
\draw[arrow, unsure] (root) to[bend right] (b1);
\draw[arrow] (a1) to (b1);
\draw[arrow] (a2) to (b2);
\draw[arrow, sure] (a2) to (b3);
\draw[arrow, medsure] (a2) to (c1);
\draw[arrow, unsure] (b2) to (c1);
\draw[arrow, verysure] (b3) to (c1);
\draw[arrow, medsure] (b3) to (c2);
\draw[arrow] (b3) to (c3);

\node[right=of a2] (l1) {};
\node[right=1cm of l1] (l2) {};
\draw[very thick, red, ->] (l1) to (l2);

\node[below=of b1] {(1)};

% \node[below=1cm of b2, anchor=south] {$G$};
\end{scope}

\begin{scope}[xshift=4cm]

\node[point] (root) at (0, 0) {};
\node[point, below left=of root] (a1) {};
\node[point, below right=of root] (a2) {};
\node[point, below left=of a1] (b1) {};
\node[point, below left=of a2] (b2) {};
\node[point, below right=of a2] (b3) {};
\node[point, below left=of b3] (c1) {};
\node[point, below=of b3] (c2) {};
\node[point, below right=of b3] (c3) {};

\draw[arrow, verysure] (root) to (a1);
\draw[arrow, sure] (root) to (a2);
% \draw[arrow, unsure] (root) to (b2);
% \draw[arrow, medsure] (root) to[bend left] (b3);
% \draw[arrow, unsure] (root) to[bend right] (b1);
\draw[arrow] (a1) to (b1);
\draw[arrow] (a2) to (b2);
\draw[arrow, sure] (a2) to (b3);
% \draw[arrow, medsure] (a2) to (c1);
\draw[arrow, unsure] (b2) to (c1);
\draw[arrow, verysure] (b3) to (c1);
\draw[arrow, medsure] (b3) to (c2);
\draw[arrow] (b3) to (c3);

\node[right=of a2] (l1) {};
\node[right=1cm of l1] (l2) {};
\draw[very thick, red, ->] (l1) to (l2);

\node[below=of b1] {(2)};
% \node[below=2.6cm of b2, anchor=south] {$G_\textrm{red}$};
\end{scope}

\begin{scope}[xshift=8cm]

\node[point] (root) at (0, 0) {};
\node[point, below left=of root] (a1) {};
\node[point, below right=of root] (a2) {};
\node[point, below left=of a1] (b1) {};
\node[point, below left=of a2] (b2) {};
\node[point, below right=of a2] (b3) {};
\node[point, below left=of b3] (c1) {};
\node[point, below=of b3] (c2) {};
\node[point, below right=of b3] (c3) {};

\draw[arrow, verysure] (root) to (a1);
\draw[arrow, sure] (root) to (a2);
% \draw[arrow, unsure] (root) to (b2);
% \draw[arrow, medsure] (root) to[bend left] (b3);
% \draw[arrow, unsure] (root) to[bend right] (b1);
\draw[arrow] (a1) to (b1);
\draw[arrow] (a2) to (b2);
\draw[arrow, sure] (a2) to (b3);
% \draw[arrow, medsure] (a2) to (c1);
%\draw[arrow, unsure] (b2) to (c1);
\draw[arrow, verysure] (b3) to (c1);
\draw[arrow, medsure] (b3) to (c2);
\draw[arrow] (b3) to (c3);

\node[below=of b1] {(3)};
% \node[below=2.6cm of b2, anchor=south] {$\hat{T}$};
\end{scope}

\end{tikzpicture}
    \end{figure}
\end{frame}

\subsection{Évaluation}

\begin{frame}{Méthode d'évaluation}
\begin{itemize}
    \item On évalue notre approche sur DBpédia (\href{dbpedia.org}{http://dbpedia.org})
    \item On utilise une version simplifiée de la taxonomie DBpédia, qui couvre 75\% des triplets de type de DBpédia
    \item Métriques pour comparer une liste d'axiomes de référence avec une liste d'axiomes extraits :
    \begin{itemize}
        \item \textbf{précision} : nombre d'axiomes extraits qui sont vrais
        \item \textbf{rappel} : nombre d'axiomes vrais qui sont extraits
        \item \textbf{mesure $F_1$} : moyenne harmonique de la précision et du rappel
    \end{itemize}
    \item Deux modes d'évaluation : direct ou transitif
    \item Modèles de plongement testés : ComplEx \cite{complex}, DistMult \cite{distmult}, RDF2Vec \cite{ristoski2016rdf2vec}, TransE \cite{bordes2013translating}
    \item Comparaison des méthodes MLI et MLM avec TIEmb \cite{ristoski2017large}
\end{itemize}
\end{frame}

\begin{frame}
\resizebox{\columnwidth}{!}{%
\begin{tabular}{|lll|ccc|ccc|ccc|}
    \hline 
                               &      &          & \multicolumn{3}{c|}{Directe}              & \multicolumn{3}{c|}{Transitive}     & \multicolumn{3}{c|}{Moyenne}     \\
        Méthode	&	Plongements	&	Distance	&	$p$	&	$r$	&	$F_1$	&	$p$	&	$r$	&	$F_1$	&	$p$	&	$r$	&	$F_1$	\\	
\hline \multirow{8}{*}{TIEmb}	&	\multirow{2}{*}{ComplEx}	&	cos	&	0.38	&	0.38	&	0.38	&	0.28	&	0.57	&	0.37	&	0.33	&	0.48	&	0.38	\\
	&		&	euc	&	0.39	&	0.42	&	0.4	&	0.34	&	0.8	&	0.48	&	0.37	&	0.61	&	0.44	\\[2pt] 
	&	\multirow{2}{*}{DistMult}	&	cos	&	0.26	&	0.27	&	0.26	&	0.19	&	0.44	&	0.27	&	0.23	&	0.36	&	0.27	\\
	&		&	euc	&	0.37	&	0.41	&	0.39	&	0.33	&	0.79	&	0.47	&	0.35	&	0.60	&	0.43	\\[2pt] 
	&	\multirow{2}{*}{RDF2Vec}	&	cos	&	0.77	&	\textbf{0.84}	&	0.81	&	0.43	&	0.87	&	0.57	&	0.60	&	0.86	&	0.69	\\
	&		&	euc	&	0.70	&	0.77	&	0.73	&	0.30	&	0.73	&	0.42	&	0.50	&	0.75	&	0.58	\\[2pt] 
	&	\multirow{2}{*}{TransE}	&	cos	&	0.77	&	0.72	&	0.74	&	0.83	&	0.89	&	0.86	&	0.80	&	0.81	&	0.80	\\
	&		&	euc	&	0.76	&	0.75	&	0.76	&	0.7	&	0.9	&	0.79	&	0.73	&	0.83	&	0.78	\\[2pt] 
\hline \multirow{8}{*}{\shortstack[c]{Liaison injective \\ (MLI)}}	&	\multirow{2}{*}{ComplEx}	&	cos	&	0.53	&	0.5	&	0.52	&	0.78	&	0.7	&	0.74	&	0.66	&	0.60	&	0.63	\\
	&		&	euc	&	0.51	&	0.44	&	0.47	&	0.87	&	0.52	&	0.65	&	0.69	&	0.48	&	0.56	\\[2pt] 
	&	\multirow{2}{*}{DistMult}	&	cos	&	0.55	&	0.47	&	0.5	&	0.73	&	0.57	&	0.64	&	0.64	&	0.52	&	0.57	\\
	&		&	euc	&	0.43	&	0.36	&	0.39	&	0.75	&	0.54	&	0.63	&	0.59	&	0.45	&	0.51	\\[2pt] 
	&	\multirow{2}{*}{RDF2Vec}	&	cos	&	0.60	&	0.44	&	0.50	&	0.88	&	0.50	&	0.63	&	0.74	&	0.47	&	0.57	\\
	&		                        &	euc	&	0.60	&	0.53	&	0.56	&	0.89	&	0.64	&	0.74	&	0.74	&	0.58	&	0.65	\\[2pt] 
	&	\multirow{2}{*}{TransE}	&	cos	&	0.82	&	0.8	&	0.81	&	0.97	&	0.89	&	\textbf{0.93}	&	\textbf{0.90}	&	0.85	&	0.87	\\
	&		&	euc	&	0.73	&	0.67	&	0.7	&	\textbf{0.99}	&	0.68	&	0.8	&	0.86	&	0.68	&	0.75	\\[2pt] 
\hline \multirow{8}{*}{\shortstack[c]{Liaison multiple \\ (MLM)}}	&	\multirow{2}{*}{ComplEx}	&	cos	&	0.52	&	0.48	&	0.5	&	0.89	&	0.7	&	0.79	&	0.71	&	0.59	&	0.65	\\
	&		                        &	euc	&	0.51	&	0.44	&	0.47	&	0.84	&	0.6	&	0.7	&	0.68	&	0.52	&	0.59	\\[2pt] 
	&	\multirow{2}{*}{DistMult}	&	cos	&	0.5	&	0.44	&	0.47	&	0.87	&	0.65	&	0.74	&	0.69	&	0.55	&	0.61	\\
	&		                        &	euc	&	0.48	&	0.42	&	0.45	&	0.86	&	0.65	&	0.74	&	0.72	&	0.37	&	0.49	\\[2pt]  
	&	\multirow{2}{*}{RDF2Vec}	&	cos	&	0.61	&	0.36	&	0.45	&	0.84	&	0.39	&	0.53	&	0.77	&	0.51	&	0.61	\\
	&		                        &	euc	&	0.61	&	0.47	&	0.53	&	0.92	&	0.55	&	0.69	&	0.74	&	0.58	&	0.65	\\[2pt] 
	&	\multirow{2}{*}{TransE}	    &	cos	&	\textbf{0.83}	&	0.81	&	\textbf{0.82}	&	0.93	&	\textbf{0.93}	&	\textbf{0.93}	&	0.88	&	\textbf{0.87}	&	\textbf{0.88}	\\
	&	  	                        &	euc	&	0.78	&	0.77	&	0.77	&	0.98	&	0.87	&	0.92	&	0.88	&	0.82	&	0.85	\\[2pt] 
\hline
    \end{tabular}  
}
\end{frame}

\section{Extraction de taxonomie expressive}

\subsection{Présentation}
\begin{frame}{Limites de l'approche précédente}
    
\end{frame}

\begin{frame}{Principe général}
    \begin{figure}
        \centering
        \begin{tikzpicture}[
  box/.style={draw=gray, thick, align=center},
  arrow/.style={->, semithick, shorten >=5pt,}
]
\def\l{2};
\node[box] (s2) at (1.5*\l, 0) {Entités à regrouper \\ $\mathcal{D}_a$};
\node[box] (s1) at (0, \l) {Axiome $a$};
\node[box] (s3) at (0, -\l) {Arbre de clustering \\$X_a$};
\node[box] (s4) at (-1.5*\l, 0) {Cluster à étiqueter \\$C$};

% \textit{Ensemble de $N$ entités vérifiant $a$}

\draw [arrow] (s1) to[out=0, in=90] node[auto] {Tirage aléatoire} (s2);
\draw [arrow] (s2) to[out=270, in=0] node[auto, align=left] {Regroupement \\ hiérarchique} (s3);
\draw [arrow] (s3) to[out=180, in=270]  node[auto] {Parcours de l'arbre}  (s4);
\draw [arrow] (s4) to[out=90, in=180]  node[auto] {Extraction d'axiomes}  (s1);

\end{tikzpicture}
    \end{figure}
\end{frame}

\subsection{Méthode proposée}

\begin{frame}{Prélèvement et regroupement}
    
\end{frame}

\begin{frame}{Extraction d'axiomes}
    
\end{frame}

\subsection{Évaluation}
\begin{frame}{Évaluation quantitative}
    
\end{frame}

\begin{frame}{Discussion sur les axiomes obtenus}
    
\end{frame}

\section{Conclusion}

\begin{frame}{Conclusion et questions}


\end{frame}

\begin{frame}
        \frametitle{Bibliographie}
        \bibliographystyle{apalike}
        \bibliography{refs.bib}
\end{frame}
\end{document}
